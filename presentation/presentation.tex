\documentclass{beamer}

\usepackage{hyperref}
\usepackage{listings}
\usepackage{color}
\usepackage{hyperref}

\usepackage[backend=bibtex]{biblatex}
\addbibresource{citations.bib}

% http://tex.stackexchange.com/questions/68080/beamer-bibliography-icon
\setbeamertemplate{bibliography item}{%
  \ifboolexpr{ test {\ifentrytype{book}} or test {\ifentrytype{mvbook}}
    or test {\ifentrytype{collection}} or test {\ifentrytype{mvcollection}}
    or test {\ifentrytype{reference}} or test {\ifentrytype{mvreference}} }
    {\setbeamertemplate{bibliography item}[book]}
    {\ifentrytype{online}
       {\setbeamertemplate{bibliography item}[online]}
       {\setbeamertemplate{bibliography item}[article]}}%
  \usebeamertemplate{bibliography item}}

\defbibenvironment{bibliography}
  {\list{}
     {\settowidth{\labelwidth}{\usebeamertemplate{bibliography item}}%
      \setlength{\leftmargin}{\labelwidth}%
      \setlength{\labelsep}{\biblabelsep}%
      \addtolength{\leftmargin}{\labelsep}%
      \setlength{\itemsep}{\bibitemsep}%
      \setlength{\parsep}{\bibparsep}}}
  {\endlist}
  {\item}

% Colours for beamer.
\setbeamercolor{frametitle}{fg=orange}
\setbeamertemplate{itemize item}{\color{orange}$\blacksquare$}
\setbeamertemplate{itemize subitem}{\color{orange}$\blacktriangleright$}

% Colours for syntax highlighting
\definecolor{syntax_red}{rgb}{0.7, 0.0, 0.0} % For strings
\definecolor{syntax_green}{rgb}{0.15, 0.5, 0.25} % For comments
\definecolor{syntax_orange}{rgb}{0.7, 0.4, 0.2} % For keywords


% Haskell settings for lstlisting
\lstset{language=C,
basicstyle=\ttfamily\tiny,
keywordstyle=\color{syntax_orange}\bfseries,
stringstyle=\color{syntax_red},
commentstyle=\color{syntax_green},
numbers=none,
numberstyle=\color{black},
stepnumber=1,
frame=single,
breaklines=true,
numbersep=10pt,
tabsize=4,
showspaces=false,
showstringspaces=false}

\author{
  Beck, Calvin\\
  \href{mailto:hobbes@seas.upenn.edu}{hobbes@seas.upenn.edu}
}

\begin{document}

\begin{frame}
  \frametitle{The Whacky World of Undefined Behaviour}
  \maketitle
\end{frame}

\section{Introduction}

\begin{frame}
  \frametitle{What is this Talk about?}

  Undefined behaviour! With a smack of LLVM.\\~\

  We'll cover things like:

  \begin{itemize}
  \item What is undefined behaviour?
  \item What happens when you encounter UB?
  \item How is UB useful? Should we avoid it?
    \begin{itemize}
    \item Optimizations?
    \end{itemize}
  \item UB in LLVM (and indeterminate values)
  \item How this all fits into Vellvm
  \end{itemize}
  \vspace{\baselineskip}
  Not for anything in particular! It's just a fun topic, and hopefully
  talking about it will clarify some things for myself and you!

\end{frame}

\section{What is undefined behaviour?}

\begin{frame}[fragile]
  \frametitle{What is undefined behaviour (UB)?}

  It's behaviour...\\~

  \pause

  That's undefined.\\~

  \pause

  \huge{Done.}

\end{frame}

\begin{frame}
  \frametitle{What is undefined behaviour (UB?)}

  Why does this seem hard?\\~

  \pause
  \begin{itemize}
\item Easy to conflate with things like implementation defined
  behaviour... Which is sort of different.
  \pause
\item Language dependent.
  \begin{itemize}
  \item  Array out of bounds in Python? Exception, not UB.
    \pause
  \item Array out of bounds in C? ... Pray.
  \end{itemize}
  \end{itemize}
\end{frame}

\begin{frame}
  \frametitle{What happens when you encounter UB?}

  \pause

  {\huge ANYTHING.}\\~

  \pause

  {\huge Yes. \pause Anything.}\\~
  
\end{frame}

\begin{frame}
  \frametitle{What happens when you encounter UB?}

  Compiler will do whatever it finds easiest or most efficient.\\~

  \begin{itemize}
  \item noop, and then continue
  \item halt
  \item halt {\bf and} catch fire
  \item erase the hard drive
    \pause
    \begin{itemize}
    \item No, seriously. Erase the hard drive.
    \item \url{https://kristerw.blogspot.com/2017/09/why-undefined-behavior-may-call-never.html}
    \end{itemize}
    \pause
  \item time travel
    \pause
    \begin{itemize}
    \item no, really.
    \item \url{https://devblogs.microsoft.com/oldnewthing/?p=633}
    \end{itemize}
    \pause
  \item nasal demons?
    \pause
    \begin{itemize}
    \item \url{https://en.wikipedia.org/wiki/Nasal\_demons}
      \pause
    \item So far I'm pretty sure this is just a joke, but I wouldn't
      rule it out.
    \end{itemize}
  \end{itemize}
\end{frame}

\begin{frame}
  \frametitle{Why is this useful?}

  Why have UB at all? Isn't it... {\tiny crazy?} \\~

  \pause

  \begin{itemize}
  \item PL without UB might have a lot of dynamic sanity checks.
    \begin{itemize}
    \item Bounds checking.
    \end{itemize}
    \pause
  \item What about type systems?
    \begin{itemize}
    \item Static checks can eliminate some dynamic checks
    \item Bounds checking still common.
    \end{itemize}
  \end{itemize}

  Instead, why not...\\~
  \pause

  Do nothing?
\end{frame}

\begin{frame}
  \frametitle{Why is this useful?}

  UB may seem somewhat unprincipled, but it has advantages:\\~

  \begin{itemize}
  \item Gives compiler an axiom.
  \item Puts burden on programmer, or other tools
  \end{itemize}

\end{frame}

\begin{frame}[fragile,fragile]
  \frametitle{UB can reflect programmer intent}

  I want to change this...\\~

\begin{lstlisting}
a + b < a + c
\end{lstlisting}

  \pause

  To this:\\~

\begin{lstlisting}
b < a
\end{lstlisting}

  After all, who wants to do 2 extra additions?\\~

  \pause

  But this is sort of wrong...\\~

\begin{lstlisting}
1 + INT_MAX < 1 + 3
// This evaluates to
INT_MIN < 4 == True

// But...
INT_MAX < 3 == False
\end{lstlisting}
\end{frame}

\begin{frame}
  \frametitle{References}

  \nocite{*}
  \printbibliography

  These are all good resources! You should look at them!
\end{frame}
\end{document}

%%% Local Variables:
%%% mode: latex
%%% TeX-master: t
%%% End:

\documentclass{beamer}

\usepackage{hyperref}
\usepackage{listings}
\usepackage{color}
\usepackage{hyperref}

%\def\lstlanguagefiles{defManSSR.tex}
%\lstloadlanguages{SSR}

\usepackage[backend=bibtex]{biblatex}
\addbibresource{citations.bib}

% http://tex.stackexchange.com/questions/68080/beamer-bibliography-icon
\setbeamertemplate{bibliography item}{%
  \ifboolexpr{ test {\ifentrytype{book}} or test {\ifentrytype{mvbook}}
    or test {\ifentrytype{collection}} or test {\ifentrytype{mvcollection}}
    or test {\ifentrytype{reference}} or test {\ifentrytype{mvreference}} }
    {\setbeamertemplate{bibliography item}[book]}
    {\ifentrytype{online}
       {\setbeamertemplate{bibliography item}[online]}
       {\setbeamertemplate{bibliography item}[article]}}%
  \usebeamertemplate{bibliography item}}

\defbibenvironment{bibliography}
  {\list{}
     {\settowidth{\labelwidth}{\usebeamertemplate{bibliography item}}%
      \setlength{\leftmargin}{\labelwidth}%
      \setlength{\labelsep}{\biblabelsep}%
      \addtolength{\leftmargin}{\labelsep}%
      \setlength{\itemsep}{\bibitemsep}%
      \setlength{\parsep}{\bibparsep}}}
  {\endlist}
  {\item}

% Colours for beamer.
\setbeamercolor{frametitle}{fg=orange}
\setbeamertemplate{itemize item}{\color{orange}$\blacksquare$}
\setbeamertemplate{itemize subitem}{\color{orange}$\blacktriangleright$}

% Colours for syntax highlighting
\definecolor{syntax_red}{rgb}{0.7, 0.0, 0.0} % For strings
\definecolor{syntax_green}{rgb}{0.15, 0.5, 0.25} % For comments
\definecolor{syntax_orange}{rgb}{0.7, 0.4, 0.2} % For keywords


% C settings for lstlisting
\lstset{language=C,
basicstyle=\ttfamily\tiny,
keywordstyle=\color{syntax_orange}\bfseries,
stringstyle=\color{syntax_red},
commentstyle=\color{syntax_green},
numbers=none,
numberstyle=\color{black},
stepnumber=1,
frame=single,
breaklines=true,
numbersep=10pt,
tabsize=4,
showspaces=false,
showstringspaces=false}

\author{
  Beck, Calvin\\
  \href{mailto:hobbes@seas.upenn.edu}{hobbes@seas.upenn.edu}
}

\begin{document}

\begin{frame}
  \frametitle{The Whacky World of Undefined Behaviour}
  \maketitle
\end{frame}

\section{Introduction}

\begin{frame}
  \frametitle{What is this Talk about?}

  Undefined behaviour! With a smack of LLVM.\\~\

  We'll cover things like:

  \begin{itemize}
  \item What is undefined behaviour?
  \item What happens when you encounter UB?
  \item How is UB useful? Should we avoid it?
    \begin{itemize}
    \item Optimizations?
    \end{itemize}
  \item UB in LLVM (and indeterminate values)
  \item How this all fits into Vellvm
  \end{itemize}
  \vspace{\baselineskip}
  Not for anything in particular! It's just a fun topic, and hopefully
  talking about it will clarify some things for myself and you!

\end{frame}

\section{What is undefined behaviour?}

\begin{frame}[fragile]
  \frametitle{What is undefined behaviour (UB)?}

  It's behaviour...\\~

  \pause

  That's undefined.\\~

  \pause

  \huge{Done.}

\end{frame}

\begin{frame}
  \frametitle{What is undefined behaviour (UB)?}

  Why does this seem hard?\\~

  \pause
  \begin{itemize}
\item Easy to conflate with things like implementation defined
  behaviour... Which is sort of different.
  \pause
\item Language dependent.
  \begin{itemize}
  \item  Array out of bounds in Python? Exception, not UB.
    \pause
  \item Array out of bounds in C? ... Pray.
  \end{itemize}
  \end{itemize}
\end{frame}

\begin{frame}
  \frametitle{What happens when you encounter UB?}

  \pause

  {\huge ANYTHING.}\\~

  \pause

  {\huge Yes. \pause Anything.}\\~
  
\end{frame}

\begin{frame}
  \frametitle{What happens when you encounter UB?}

  Compiler will do whatever it finds easiest or most efficient.\\~

  \begin{itemize}
  \item noop, and then continue
  \item halt
  \item halt {\bf and} catch fire
  \item erase the hard drive
    \pause
    \begin{itemize}
    \item No, seriously. Erase the hard drive.
    \item \url{https://kristerw.blogspot.com/2017/09/why-undefined-behavior-may-call-never.html}
    \end{itemize}
    \pause
  \item time travel
    \pause
    \begin{itemize}
    \item no, really.
    \item \url{https://devblogs.microsoft.com/oldnewthing/?p=633}
    \end{itemize}
    \pause
  \item nasal demons?
    \pause
    \begin{itemize}
    \item \url{https://en.wikipedia.org/wiki/Nasal\_demons}
      \pause
    \item So far I'm pretty sure this is just a joke, but I wouldn't
      rule it out.
    \end{itemize}
  \end{itemize}
\end{frame}

\begin{frame}
  \frametitle{Why is this useful?}

  Why have UB at all? Isn't it... {\tiny crazy?} \\~

  \pause

  \begin{itemize}
  \item PL without UB might have a lot of dynamic sanity checks.
    \begin{itemize}
    \item Bounds checking.
    \end{itemize}
    \pause
  \item What about type systems?
    \begin{itemize}
    \item Static checks can eliminate some dynamic checks
    \item Bounds checking still common.
    \end{itemize}
  \end{itemize}

  Instead, why not...\\~
  \pause

  Do nothing?
\end{frame}

\begin{frame}
  \frametitle{Why is this useful?}

  UB may seem somewhat unprincipled, but it has advantages:\\~

  \begin{itemize}
  \item Gives compiler an axiom.
  \item Puts burden on programmer, or other tools
  \end{itemize}

\end{frame}

\begin{frame}[fragile,fragile]
  \frametitle{UB can reflect programmer intent}

  I want to change this...\\~

\begin{lstlisting}
a + b < a + c
\end{lstlisting}

  \pause

  To this:\\~

\begin{lstlisting}
b < a
\end{lstlisting}

  After all, who wants to do 2 extra additions?\\~

  \pause

  But this is sort of wrong...\\~

\begin{lstlisting}
1 + INT_MAX < 1 + 3
// This evaluates to
INT_MIN < 4 == True

// But...
INT_MAX < 3 == False
\end{lstlisting}
\end{frame}

\begin{frame}[fragile]
  \frametitle{Pointer aliasing --- optimizations with undecidability}

  Undefined behavour allows compilers to make optimizations based on
  undecidable invariants.\\~

  \pause

  It might be nice to optimize this:

\begin{lstlisting}
void sum(double *total, double *array, size_t len )
{
    *total = 0;
    for (size_t i=0; i<len; i++) {
        *total += array[i];
    }
}
\end{lstlisting}

  To this:

\begin{lstlisting}
void sum(double *total, double *array, size_t len )
{
    double local_total = 0;
    for (size_t i=0; i<len; i++) {
        local_total += array[i];
    }

    *total = local_total;
}
\end{lstlisting}

\end{frame}

\begin{frame}[fragile]
  \frametitle{C's restrict keyword}

\begin{lstlisting}
void sum(double* restrict total, double* restrict array, size_t len )
{
    *total = 0;
    for (size_t i=0; i<len; i++) {
        *total += array[i];
    }
}
\end{lstlisting}

  Allows optimization to

\begin{lstlisting}
void sum(double *total, double *array, size_t len )
{
    double local_total = 0;
    for (size_t i=0; i<len; i++) {
        local_total += array[i];
    }

    *total = local_total;
}
\end{lstlisting}

  Because restrict says ``these pointers don't alias with
  anything''. If they happen to alias, then it's UB, and the program
  can do whatever.

\end{frame}

\begin{frame}[fragile,fragile]
  \frametitle{WHERE WE'RE GOING WE DON'T NEED ROADS}

  \pause
  So, how powerful is undefined behaviour?\\~

  \pause
  Can we optimize this:

\begin{lstlisting}
char inp = getchar();
if ('A' == inp) {
    printf("Hello, world!\n");
    x = 1 / 0;
}
\end{lstlisting}

  \pause

  To this?

\begin{lstlisting}
getchar();
\end{lstlisting}

  \pause

  Or maybe just this?

\begin{lstlisting}
char inp = getchar();
if ('A' == inp) {
    printf("Hello, world!\n");
}
\end{lstlisting}

  \pause

  Does UB mean {\bf cannot} happen or {\bf anything} can happen?

\end{frame}

\begin{frame}[fragile]
  \frametitle{Depends who you ask}

  Supposedly the C++ standard says this:\\~

\begin{verbatim}
However, if any such execution contains an undefined operation,
this International Standard places no requirement on the
implementation executing that program with that input
(not even with regard to operations preceding the
first undefined operation).
\end{verbatim}

  \pause

  So, C++ seems to explicitly allow this kind of time travel.\\~

  In general UB seems to be somewhat of an underspecified
  concept. Both options seem reasonable, depending on what you want to
  do.

\end{frame}

\begin{frame}
  \frametitle{Undefined behaviour in IR}

  Now we know a few things about UB.\\~

  \begin{itemize}
  \item Seems to be useful for optimizations.
  \item With languages like C, passes burden onto the programmer :(.
  \end{itemize}

  Whether or not you like UB in a programming language... It seems to
  make some sense for an IR.

  \begin{itemize}
  \item Again, optimizations
  \item Way to pass down invariants from higher levels
    \begin{itemize}
    \item Either the programmer proves something (or pretends to have
      proved something)
    \item Type systems when the types get erased
    \item Other external tools / checkers
      \begin{itemize}
      \item Maybe you even use something like Coq to verify your
        C. Why not get better optimizations in that case?
      \end{itemize}
    \end{itemize}
  \end{itemize}
\end{frame}

\begin{frame}[fragile]
  \frametitle{Expressive UB in IRs?}

  Why not have a lot of control over exactly what counts as UB?\\~

  \pause

  LLVM Has some of this already:

\begin{verbatim}
<result> = add <ty> <op1>, <op2>
<result> = add nuw <ty> <op1>, <op2>
<result> = add nsw <ty> <op1>, <op2>
<result> = add nuw nsw <ty> <op1>, <op2>
\end{verbatim}

  Lets compilers pick the option that best suits the situation.
\end{frame}

\begin{frame}
  \frametitle{Indeterminate values in LLVM}

  LLVM has indenterminate values in the form of {\tt undef} and {\tt
    poison}.\\~

  \begin{itemize}
  \item Not undefined behaviour themselves
    \begin{itemize}
    \item But, closely related
    \end{itemize}
  \item ``it doesn't matter what value this has''
  \item ``this value won't be used''
  \end{itemize}

  \pause

  {\tt undef} and {\tt poison} are sort of messy. We're doing our best
  to describe what we believe LLVM to be in Vellvm, but it's not
  always clear with these two.

\end{frame}

\begin{frame}
  \frametitle{Undef}

  {\tt undef} is more like ``undefined value'' than ``undefined
  behaviour''.\\~

  \begin{itemize}
  \item Unspecified / arbitrary bit pattern
    \begin{itemize}
    \item uninitialized variables
    \item loads from uninitialized memory
    \end{itemize}
  \end{itemize}
\end{frame}

  \begin{frame}[fragile]
    \frametitle{Undef's laziness}

    {\tt undef} is a little weird.\\~

\begin{lstlisting}
%x = i32 undef
%y = add i32 %x %x
\end{lstlisting}

    Is the same as this:\\~

\begin{lstlisting}
%x = i32 undef
%y = i32 undef  ;; Not just even numbers
\end{lstlisting}

    Why?\\~

    \pause

    {\tt undef} isn't just ``I don't care what value this
    variable has'' it's more like ``I don't even care if this variable
    has a specific value.''\\~

    No need to save an arbitrary bit pattern if it shouldn't change
    the behaviour of the program.
  \end{frame}

  \begin{frame}[fragile]
    \frametitle{Undef gets a little weirder...}

    But \texttt{undef} does not always beget undef...\\~

\begin{lstlisting}
%x = i32 undef
%y = mul i32 %x 2
\end{lstlisting}

    What's {\tt y}? \\~

    \pause

    It's also an indeterminate value. The set of all even numbers,
    and...

\begin{lstlisting}
%z = add i32 %y %y
\end{lstlisting}

    \pause

    Also the set of even numbers. Not some multiple of 4. LLVM again allows
    {\tt y} to have a different value each time, but it must be a
    multiple of 2.
  \end{frame}

\begin{frame}[fragile]
  \frametitle{How is undef useful?}

\begin{lstlisting}
int x;
if (cond) {
    x = f();
}

if (cond2) {
    g(x);
}
\end{lstlisting}

  Why force a value for {\tt x}? What if $\mathrm{cond2}
  \implies \mathrm{cond}$?\\~

  We can assume {\tt x} is assigned before use, and avoid an arbitrary
  assignment to {\tt x}.

\end{frame}

\begin{frame}[fragile]
  \frametitle{undef under Vellvm}

  Keep full {\tt undef} expressions to avoid early concretization.\\~

\begin{lstlisting}
(undef + 2) * (3 * undef)  // Left in this complicated form
\end{lstlisting}

  Makes interpreter / model much more complicated. \\~

  We have ``pick'' events whenever we have to concretize {\tt undef},
  when side-effecting operations occur. Even something like {\tt div}
  forces pick events.

\end{frame}

\begin{frame}
  \frametitle{Spreading poison}

  Sometimes we need something a little stronger than {\tt undef}:
  {\tt poison}.\\~

  \begin{itemize}
  \item Can always relax {\tt poison} to {\tt undef}.
  \item Poison is much simpler.
  \item Operations on poison result in poison
  \item If you use poison with something that causes side-effects you
    get UB.
  \end{itemize}
\end{frame}

\begin{frame}[fragile]
  \frametitle{How can we use poison?}

  ~poison~ is really useful as a kind of deferred undefined
  behaviour. Can use it for a kind of speculative execution.

\begin{lstlisting}
      for (size_t i = 0; i <= n; ++i) {
          a[i] = x + 1;
      }
\end{lstlisting}

  Can we optimize this? \pause Maybe to this?

\begin{lstlisting}
      int y = x + 1;
      for (size_t i = 0; i <= n; ++i) {
          a[i] = y;
      }
\end{lstlisting}

  Want to delay UB from $x + 1$, so LLVM gives overflows the value of
  poison.
  
\end{frame}

\begin{frame}[fragile]
  \frametitle{More uses of poison}
  Similarly, can we optimize this?
  
\begin{lstlisting}
      for (int i = 0; i <= n; ++i) {
          a[(size_t)i] = 42;
      }
\end{lstlisting}

  \pause To this?

\begin{lstlisting}
      for (size_t i = 0; i <= n; ++i) {
          a[i] = 42;
      }
\end{lstlisting}

\end{frame}

\begin{frame}
  \frametitle{Poisoning Vellvm}

  Poison in Vellvm is much simpler than {\tt undef}. Basically just
  have operations return a {\tt poison} if they're given {\tt
    poison}.\\~

  Certain operations on poison will trigger UB.
  
\end{frame}

\begin{frame}[fragile]
  \frametitle{poison vs undef}

  \begin{itemize}
  \item Don't want to get too hung up on this, because it is
    confusing.
  \item {\tt poison} and {\tt undef} are similar, but  slightly
    different
  \item  Justify different optimizations

\begin{lstlisting}
a + 1 > a
\end{lstlisting}

    To...

\begin{lstlisting}
true // Or I guess 1 in C...
\end{lstlisting}
  \end{itemize}

  only works with poison for $a + 1$ overflowing.
\end{frame}

\begin{frame}[fragile]
  \frametitle{Refinement and Vellvm}

\begin{lstlisting}
    (* Refinement relation for uvalues *)
    Inductive refine_uvalue: uvalue -> uvalue -> Prop :=
    | UndefPoison: forall t, refine_uvalue (UVALUE_Undef t) UVALUE_Poison
    | RefineConcrete: forall uv1 uv2, (forall dv, concretize uv1 dv -> concretize uv2 dv) -> refine_uvalue uv1 uv2
    .
\end{lstlisting}

  \pause
  
\begin{lstlisting}
    (* Refinement of uninterpreted mcfg *)
    Definition refine_L0: relation (itree L0 uvalue) := eutt refine_uvalue.

    (* Refinement of mcfg after globals *)
    Definition refine_res1 : relation (global_env * uvalue)
      := TT x refine_uvalue.

    (* ... *)
\end{lstlisting}

  \pause

\begin{lstlisting}
      (* Refinement for after interpreting pick. *)
    Definition refine_L4 : relation ((itree L4 (memory * (local_env * stack * (global_env * uvalue)))) -> Prop)
      := fun ts ts' => forall t, ts t -> exists t', ts' t' /\ eutt refine_res3 t t'.
\end{lstlisting}

\end{frame}

\begin{frame}
  \frametitle{Better future}

  More clear semantics for undef / poison? Freeze thaw?

  \begin{itemize}
  \item remove undef. Only poison.
  \item freeze of poison is nondeterministic choice.
  \end{itemize}
\end{frame}

  \begin{frame}[allowframebreaks]
    \frametitle{References}

  \nocite{*}
  \printbibliography

  These are all good resources!
\end{frame}
\end{document}

%%% Local Variables:
%%% mode: latex
%%% TeX-master: t
%%% End:
